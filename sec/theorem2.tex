% !TEX root = ../cobar2.tex

The goal of this section is to prove \cref{t:2nd main thm in the intro}.
Explicitly, we extend the $E_{\infty}$-coalgebra structure on the cobar construction given by \cref{l:lift of cobar to e-infty} to the extended cobar construction of \cite{hess2010cobar} (\cref{l:AhatUM}).
Then, we relate the extended cobar construction and the simplicial chains on the Kan loop group as $E_{\infty}$-coalgebras (\cref{l:AhatandGX}).

The first goal is achieved by interpreting the extended cobar construction as a monoidal cubical set (\cref{ss:extended cubical cobar}).
The second goal is obtained by using the rigidification functor (\cref{ss:rigidification}) as a middle object.
The rigidification functor is naturally isomorphic to the triangulation of the cubical cobar construction (\cref{Candgcobar}).
Furthermore, a localized version of the rigidification functor agrees, up to homotopy, with the Kan loop group (\cref{CandG}).
This fact relies on classical results from homotopy theory.
Finally, \cref{t:2nd main thm in the intro} follows by applying the corresponding functors of normalized chains and comparing these via the Cartan--Serre map, as explained in the last three subsections.