% !TEX root = ../cobar2.tex

\section{Extended cobar construction}

%The proof that Adams' map in \cref{t:1st main thm in the intro} is a quasi-isomorphism relies on the fibrancy of the singular complex.
%In general, for an arbitrary reduced simplicial set $X$, the cobar construction $\cobar\schainsA(X)$ is not quasi-isomorphic to $\cSchains(\loops|X|)$, the singular cubical chains on the based loop space of the geometric realization of $X$.
%This is true, however, if $X$ is $1$-connected.

In \cite{hess2010cobar}, it is described how to formally invert the elements of $X_1$ inside $\cobar \schainsA(X)$ to obtain the correct quasi-isomorphism type.
The resulting model, called the \textit{extended cobar construction}, is then compared, as an algebra, to the chains on the Kan loop group construction $\kan(X)$.
We now review these constructions and
%, in \cref{ss:extended cubical cobar},
reinterpret them in categorical terms.

\subsection{Extended cobar construction}

Denote the composition of the functor of chains and the cobar construction by
\[
\adams \defeq \cobar \schainsA \colon \sSet^0 \to \Mon_{\Ch},
\]
and define the functor
\[
\adamsE \colon \sSet^0 \to \Mon_{\Ch}
\]
to be the localization of the set of $0$-cycles
\[
A_X = \{ [\overline{\sigma}]+1_\k \mid \sigma \in X_1 \} \subset \adams(X)_0
\]
in the associative algebra $\adams(X)$.

The algebra $\adamsE(X)$ coincides with Hess--Tonk's extended cobar construction applied to the coalgebra $\schains(X)$ at the $\k$-basis of the degree $1$-elements given by the set $X_1$.
More precisely, using the notation of \cite{hess2010cobar}, we have $\adamsE(X) = \widehat{\cobar} \schains(X)$, where the fixed basis in the latter is given the set $X_1$.
Note that the extended cobar construction is not a functorial construction with respect to maps in $\coAlg$ since it depends on a choice of basis of the degree one $\k$-module in the underlying coalgebra.
It is, however, functorial with respect to maps of reduced simplicial sets and that is how we have interpreted it.

Furthermore, for any reduced simplicial set $X$, the natural quasi-isomorphism of algebras
\[
\phi_X \colon \adamsE(X) \to \schains(\kan(X))
\]
constructed in \cite{hess2010cobar} induces a natural isomorphism
\[
H_0(\adamsE(X)) \cong \k[\pi_1(X)].
\]

\subsection{Extended cubical cobar construction} \label{ss:extended cubical cobar}

We now want to lift the localization procedure used to define $\adamsE$ to the level of monoidal cubical sets.

We define a functor
\[
\ncobarE \colon \sSet^0 \to \Mon_{\nSet}
\]
by formally adding inverses for every $0$-dimensional necklace $\sigma \colon \simplex^1 \to X$ in $\ncobar(X)$ subject to the usual relations together with the higher degenerate necklaces generated by these new formal elements.

Define the \textit{extended cubical cobar construction}
\[
\ccobarE \colon \sSet^0 \to \Mon_{\cSet}
\]
to be the composition of functors $\ccobarE = \cP_{!} \ncobarE$.

As an immediate consequence of \cref{l:ccobar and cobar}, we obtain the following isomorphism after localizing.

\begin{corollary}
	There is a natural isomorphism of functors
	\[
	\cchains \ccobarE \, \cong \adamsE \colon \sSet^0 \to \Mon_{\Ch}.
	\]
\end{corollary}

\begin{proof}
	The isomorphism $\varphi_X \colon \cchains(\ccobar(X))\to \adams(X)$
	of \cref{l:ccobar and cobar} induces a bijection between the sets
	$X_1$ and $A_X$.
	Hence, the result follows directly from the definitions of $\ccobarE(X)$ and $\adamsE(X)$, where the sets $X_1$ and $A_X$, respectively, have been formally inverted.
\end{proof}