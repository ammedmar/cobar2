% !TEX root = ../cobar2.tex

\section{Introduction}

The groundbreaking work of Mandell has revealed that the entire homotopy type of a nilpotent and finite type space $\fX$ can be represented by the quasi-isomorphism type of its complex of singular chains, endowed with an $E_\infty$-coalgebra structure \cite{mandell2006homotopy_type}.

In a recent study \cite{medina2021cobar}, we demonstrated that for any pointed space $\fX$, the Adams' cobar construction on its singular chains, as well as the cubical singular chains of its loop space, can be endowed with a monoidal $E_\infty$-coalgebra structure. Moreover, we have confirmed that the comparison of these structures as monoidal chain complexes through Adams' map respects their $E_\infty$-structures.

In this work, our focus is on the chains of the Kan loop group \cite{bibid}\todo{Original referece?} and the extended cobar construction of Hess and Tonks \cite{hess2010cobar}.
The Kan loop group construction is a functor
\[
\kan \colon \sSet^0 \to \sGrp
\]
from reduced simplicial sets to simplicial groups, which has the property that for any reduced simplicial set $X$, the geometric realization $\bars{\kan(X)}$ is homotopy equivalent, as a topological monoid, to the based loop space $\loops\bars{X}$ \cite{berger1995loops}.
Moreover, $\kan$ and its right adjoint, a combinatorial model for the classifying space, define a Quillen equivalence.
The chains of $\kan(X)$ form a monoid with product
\[
\chains(\kan X) \ot \chains(\kan X) \xra{\EZ} \chains(\kan X \times \kan X) \to \chains(\kan X)
\]
defined using the Eilenberg--Zilber map.

To motivate the construction of Hess and Tonks, let us first define the functor $\adams$ as the composition $\cobar \circ \chains \colon \sSet^0 \to \Mon_\Ch$ of the functor of chains and the usual Adams' cobar construction.
The monoids $\adams(X)$ and $\chains(\kan X)$ are not quasi-isomorphic in general.
This can be seen from the fact that $\rH_0(\kan X)$ is always isomorphic to $ \k[\pi_1 |X|]$, whereas $\rH_0(\adams X)$ need not be a group ring \todo{Do you know of an example in the literature?}.
To resolve this issue, Hess and Tonks introduced a localized version of the cobar construction
\[
\adamsE \colon \sSet^0 \to \Mon_\Ch
\]
and a natural quasi-isomorphism
\[
\HT \colon \adamsE(X) \to \chains(\kan X)
\]
of monoids in $\Ch$.
Their functor is obtained by composing $\adams$ with a localization functor that formally inverts a basis of $\adams(X)_0$.

The first contribution of this work is the effective construction of a natural monoidal $E_\infty$-coalgebra structure on $\adamsE(X)$, which extends the monoidal coalgebra structure defined by Franz \cite{franz2020szczarba}.
Let us now turn to the comparison between $\adamsE(X)$ and $\chains(\kan X)$.

Regarding the comparison between $\adamsE(X)$ and $\chains(\kan X)$, the chains of any simplicial set are equipped with a monoidal coalgebra structure defined by the Alexander--Whitney diagonal, which can be explicitly extended to an $E_\infty$-coalgebra structure. However, since $\EZ$ is a morphism of coalgebras but not of $E_\infty$-coalgebras, $\chains(\kan X)$ is a monoid in the category of coalgebras but not in that of $E_\infty$-coalgebras.

The second contribution of this work is a new proof of a theorem of Franz, which shows that the comparison map $\HT \colon \adamsE(X) \to \chains(\kan X)$ respects the coalgebra structure. Additionally, we provide an example to demonstrate that the $E_\infty$-coalgebra structure is not necessarily preserved.

Our third contribution is the effective construction of a zig-zag of quasi-isomorphisms of $E_\infty$-coalgebras that connect $\adamsE(X)$ and $\chains(\kan X)$.
To accomplish this, we utilize the Cartan--Serre map and draw upon concepts from the theory of rigidification of $(\infty,1)$-categories.\todo{Say more if you want.}