% !TEX root = ../cobar2.tex

%The first map in this composition preserves the coassociative coalgebra structures but not the $\UM$-coalgebra structures.
%Nevertheless, in the following sections, we describe a zig-zag of quasi-isomorphisms of $E_{\infty}$-coalgebras between $\cchains(\ccobarE(X))$ and $\schains(\triangulate\ccobarE(X))$.

\section{\pdfEinfty-structures}

\TBW

\subsection{$\cM$-bialgebra}\label{ss:e-infty extension}

An \textit{$\cM$-bialgebra} is a counital coalgebra $(C, \copr, \aug)$ together with a degree $1$ linear map $\pr \colon C \ot C \to C$ satisfying
\[
\bd (a \ast b) - \bd a \ast b + (-1)^{\bars{a}} a \ast \bd b =
\aug(a) b - \aug(b) a,
\]
\[
\aug (a \pr b) = 0,
\]
for all $a, b \in C$.
As proven in \cite{medina2020prop1}, the collection of all maps $\set{C \to C^{\ot r}}_{r \in \N}$ generated by $\copr$, $\aug$ and $\pr$ make $C$ into an $E_\infty$-coalgebra, that is to say, a coalgebra over certain operad $\UM$ that is a cofibrant resolution of the terminal operad.

As proven in \cite{medina2021cobar}, the counital coalgebra structure on the tensor product of two $\cM$-bialgebras $C$ and $C'$ can be naturally extended to an $\cM$-bialgebra structure using
\begin{equation}\label{eq:monoidal_product}
	(a \ot b) \ast (c \ot d) =
	a \aug(c) \ot (b \ast d) + (a \ast c) \ot \aug(b) d,
\end{equation}
for any $a,c \in C$ and $b,d \in C'$.

For any integer $n$, the \textit{join product} $\ast \colon \chains(\simplex^n)^{\ot 2} \to \chains(\simplex^n)$ is the natural degree~$1$ linear map defined by
\begin{equation*}
	\left[v_0, \dots, v_p \right] \pr \left[v_{p+1}, \dots, v_q\right] =
	\begin{cases} (-1)^{p} \sign(\pi) \left[v_{\pi(0)}, \dots, v_{\pi(q)}\right] & \text{ if } v_i \neq v_j \text{ for } i \neq j, \\
		\hfil 0 & \text{ if not}, \end{cases}
\end{equation*}
where $\pi$ is the permutation that orders the vertices.
It is proven in \cite{medina2020prop1} that the Alexander--Whitney counital coalgebra structure together with the join product make $\chains(\simplex^n)$ into a natural $\cM$-bialgebra and, consequently, into a natural $E_\infty$-coalgebra.
We mention that this structure is induced by one preset at the level of geometric realizations \cite{medina2021prop2}.

Using the tensor product structure, we deduce a natural $\cM$-bialgebra structure on the chains of representable cubical sets
\[
\chains\big(\cube^n\big) =
\chains(\simplex^1) \ot\dotsb\ot \chains(\simplex^1),
\]
and, consequently, a natural $E_\infty$-coalgebra structure, which extends along the Yoneda inclusion to the chains on any cubical set $X$.

Since the $\EZ$ map is a morphism of counital coalgebras, we conclude that $\HT$ is as well.
As shown in \cite[$\S$5.4]{medina2022cube_einfty}, $\EZ$ does not preserve the $E_\infty$-coalgebra structure, so we will need more work to compare $\adamsE(X)$ and $\schains(\kan(X))$ as $E_\infty$-coalgebras.
We now turn to this endeavor.

\subsection{Cartan--Serre map}

In \cite[p.442]{serre1951homologie}, the cellular map $\cs \colon \gcube^n \to \gsimplex^n$ sending
$(x_1, \dots, x_n)$ to
\[
(x_1,\ x_1 x_2, \, \dots \, , \ x_1 x_2 \dotsm x_n)
\]
was used to define a natural quasi-isomorphism of coalgebras $\sSchains(\fX) \to \cSchains(\fX)$ for any topological space $\fX$.
As proven in \cite[\S5.7]{medina2022cube_einfty}, this chain map factors as a composition
\[
\schains(\sSing(\fX)) \xra{\CS_{\sSing(\fX)}}
\cchains(\cU \sSing(\fX)) \to
\cchains(\cSing(\fX)),
\]
where the first chain map, referred to as the \textit{Cartan--Serre map}, is a quasi-isomorphism defined for any simplicial set, and the second chain map is induced from a cubical map.

The Cartan--Serre map $\CS_X$ is not a morphism of $\UM$-coalgebras in general (\cite[$\S$5.4]{medina2022cube_einfty}), but it does preserve an $E_\infty$-coalgebra structure generated by a subset of operations, which we refer to as a \textit{restricted $\UM$-coalgebra} (\cite[\S5.6]{medina2022cube_einfty}).
The generating cooperations are of the form
\[
(\pr^{k_1} \ot\dotsb\ot \pr^{k_r}) \circ \tau \circ \copr^{k},
\]
where $\tau$ is the inverse of a $(k_1+1,\dots,k_r+1)$-shuffle permutation with $k_1+\dots+k_r+r = k-1$, and $\pr^{k_i}$ is defined recursively by
\[
\pr^0 = \id, \quad \pr^1 = \pr, \quad \pr^{k_i} = \pr \circ (\pr^{k_i-1} \ot\, \id).
\]
Combining this map with the unit $\xi$ of the adjuntion $(\triangulate, \cU)$ we have the following.

\begin{lemma}
	For any cubical set $Y$,
	\[
	\cchains(Y) \xra{\cchains(\xi_Y)}
	\cchains(\cubify{\triangulate Y})
	\xla{\CS_{\cT Y}} \schains(\triangulate Y),
	\]
	is a natural zig-zag of quasi-isomorphisms of restricted $\UM$-coalgebras.
\end{lemma}

\subsection{The functor $\adamsE_{\UM}$} \label{s:ahatum}

Define the functor
\[
\adamsEA \colon \sSet^0 \to \Mon_{\coAlg}
\]
as the composition
\[
\adamsEA = \cchainsA \ccobarE.
\]
Similarly define
\[
\adamsE_{\UM} \colon \sSet^0 \to \Mon_{\coAlg_{\UM}}
\]
as the composition
\[
\adamsE_{\UM} = \cchainsUM \ccobarE.
\]

It follows directly from the definitions that $\adamsEA$ is a lift of $\adamsE$, and $\adamsE_{\UM}$ is a lift $\adamsEA$.
This discussion constitutes the first half of \cref{t:2nd main thm in the intro}, which we collect in the following.

\begin{lemma} \label{l:AhatUM}
	The functor $\adamsEUM \colon \sSet^0 \to \Mon_{\coAlg_\UM}$ fits into a commutative diagram
	\[
	\begin{tikzcd}[row sep=small]
		& \Mon_{\coAlg_\UM} \arrow[d] \\
		& \Mon_{\coAlg} \arrow[d] \\
		\sSet^0
		\arrow[r, "\adamsE"', ]
		\arrow[ru, "\adamsEA"', out=45, in=180]
		\arrow[ruu, "\adamsEUM", out=90, in=180]
		& \Mon_{\Ch}.
	\end{tikzcd}
	\]
\end{lemma}


\subsection{The extended cobar construction and the chains on the Kan loop group}

Applying \cref{l:zigzag} to $Y = \ccobarE(X)$, we obtain that $\schainsUSL (\triangulate \ccobarE(X))$ and $\cchainsUSL(\ccobarE(X)) = \adamsE_{\USL}(X)$ are quasi-isomorphic via a zig-zag of natural maps of $\USL$-coalgebras.
By \cref{widehatgcobarandG} we have a weak equivalence of simplicial monoids
\[
\triangulate \ccobarE(X) \xrightarrow{\simeq} \kan(X)
\]
for any reduced simplicial set $X$.
By naturality, we have an induced quasi-isomorphism of $\USL$-coalgebras
\[
\schainsUSL \big( \triangulate \ccobarE(X) \big) \xra{\simeq}
\schainsUSL (\kan(X)).
\]
Therefore, $\adamsE_{\USL}(X)$ and $\schainsUSL(\kan(X))$ are quasi-isomorphic through a zig-zag of natural maps of $\USL$-coalgebras, as desired.
We summarize this discussion in the following lemma, constituting the second half of \cref{t:2nd main thm in the intro}.

\begin{lemma}\label{l:AhatandGX}
	For any reduced simplicial set $X$, there is a zig-zag of natural quasi-isomorphisms of $\USL$-coalgebras between $\adamsE_\USL(X)$ and $\schainsUSL(\kan(X))$.
\end{lemma}