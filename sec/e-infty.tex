% !TEX root = ../cobar2.tex

%The first map in this composition preserves the coassociative coalgebra structures but not the $\UM$-coalgebra structures.
%Nevertheless, in the following sections, we describe a zig-zag of quasi-isomorphisms of $E_{\infty}$-coalgebras between $\cchains(\ccobarE(X))$ and $\schains(\triangulate\ccobarE(X))$.

\section{\pdfEinfty-structures}\label{s:e-infty}

The goal of this section is to construct, for any reduced simplicial set, a natural zig-zag of quasi-isomorphisms relating its extended cobar construction and the chains on its Kan loop group as $E_\infty$-coalgebras.

\subsection{$\cM$-bialgebras and $\UM$-coalgebras}\label{ss:e-infty extension}

An \textit{$\cM$-bialgebra} is a counital coalgebra $(C, \copr, \aug)$ together with a degree $1$ linear map $\pr \colon C \ot C \to C$ satisfying
\[
\bd (a \ast b) - \bd a \ast b + (-1)^{\bars{a}} a \ast \bd b =
\aug(a) b - \aug(b) a,
\]
\[
\aug (a \pr b) = 0,
\]
for all $a, b \in C$.
As proven in \cite{medina2020prop1}, the collection of all maps $\set{C \to C^{\ot r}}_{r \in \N}$ generated by $\copr$, $\aug$ and $\pr$ make $C$ into an $E_\infty$-coalgebra, that is to say, a coalgebra over certain operad $\UM$ that is a cofibrant resolution of the terminal operad.

As proven in \cite{medina2021cobar}, the counital coalgebra structure on the tensor product of two $\cM$-bialgebras $C$ and $C'$ can be naturally extended to an $\cM$-bialgebra structure using
\begin{equation}\label{eq:monoidal_product}
	(a \ot b) \ast (c \ot d) =
	a \aug(c) \ot (b \ast d) + (a \ast c) \ot \aug(b) d,
\end{equation}
for any $a,c \in C$ and $b,d \in C'$.

\begin{example*}
	For any integer $n$, the \textit{join product} $\ast \colon \chains(\simplex^n)^{\ot 2} \to \chains(\simplex^n)$ is the natural degree~$1$ linear map defined by
	\begin{equation*}
	\left[v_0, \dots, v_p \right] \pr \left[v_{p+1}, \dots, v_q\right] =
	\begin{cases} (-1)^{p} \sign(\pi) \left[v_{\pi(0)}, \dots, v_{\pi(q)}\right] & \text{ if } v_i \neq v_j \text{ for } i \neq j, \\
	\hfil 0 & \text{ if not}, \end{cases}
	\end{equation*}
	where $\pi$ is the permutation that orders the vertices.
	It is proven in \cite{medina2020prop1} that the Alexander--Whitney counital coalgebra structure together with the join product make $\chains(\simplex^n)$ into a natural $\cM$-bialgebra and, consequently, into a natural $\UM$-coalgebra, which extends along the Yoneda inclusion to the chains on any simplicial set.
	We mention that this structure is induced by one preset at the level of geometric realizations \cite{medina2021prop2}.
\end{example*}

\begin{example*}
	Using Formula \eqref{eq:monoidal_product}, we deduce a natural $\cM$-bialgebra structure on the chains of representable cubical sets
	\[
	\chains\big(\cube^n\big) =
	\chains(\simplex^1) \ot\dotsb\ot \chains(\simplex^1),
	\]
	and, consequently, a natural $\UM$-coalgebra structure, which extends along the Yoneda inclusion to the chains on any cubical set as a monoidal functor (\cite[Theorem~5]{medina2021cobar}).
\end{example*}

Since for any cubical set $Y$, the Eilenberg--Zilber map $\EZ \colon \cchains(Y) \to \schains(\triangulate Y)$ is a coalgebra map, we deduce from Factorization~\eqref{e:factorization of HT} that the Hess--Tonks map $\HT \adamsE(X) \to \schains(\kan(X))$ is as well.
A result proven first by Franz \cite{franz2020szczarba}.
\todo{Is this attribution correct?}
Regarding the higher structure we have the following.

\begin{example*}
	\TBW\todo{An example using \cite[$\S$5.4]{medina2022cube_einfty} can be surely constructed.}
\end{example*}

It is not surprising that $\HT$ does not respect higher coproducts since $\EZ$ does not either.
In the next subsection we review a map $\CS \colon \schains(X) \to \cchains(\cubify X)$ that does preserve an $E_\infty$-coalgebra extension for every simplicial set $X$.

\subsection{Cartan--Serre map}

In \cite[p.442]{serre1951homologie}, the cellular map $\gcube^n \to \gsimplex^n$ sending
$(x_1, \dots, x_n)$ to
\[
(x_1,\ x_1 x_2, \, \dots \, , \ x_1 x_2 \dotsm x_n)
\]
was used to define a natural quasi-isomorphism of coalgebras $\sSchains(\fX) \to \cSchains(\fX)$ for any topological space $\fX$.
As proven in \cite[\S5.7]{medina2022cube_einfty}, this chain map factors as a composition
\[
\schains(\sSing(\fX)) \xra{\CS}
\cchains(\cU \sSing(\fX)) \to
\cchains(\cSing(\fX)),
\]
where the first chain map, referred to as the \textit{Cartan--Serre map}, is a quasi-isomorphism defined for any simplicial set, and the second chain map is induced from a cubical map.

The Cartan--Serre map $\CS$ is not a morphism of $\UM$-coalgebras in general (\cite[$\S$5.4]{medina2022cube_einfty}), but it does preserve an $E_\infty$-coalgebra structure generated by cooperations of the form
\[
(\pr^{k_1} \ot\dotsb\ot \pr^{k_r}) \circ \tau \circ \copr^{k},
\]
where $\tau$ is the inverse of a $(k_1+1,\dots,k_r+1)$-shuffle permutation with $k_1+\dots+k_r+r = k-1$, and $\pr^{k_i}$ is defined recursively by
\[
\pr^0 = \id, \quad \pr^1 = \pr, \quad \pr^{k_i} = \pr \circ (\pr^{k_i-1} \ot\, \id).
\]
We refer to such an $E_\infty$-coalgebra as a \textit{restricted $\UM$-coalgebra} (\cite[\S5.6]{medina2022cube_einfty}).

Combining the Cartan--Serre map with the unit $\xi$ of the adjuntion $(\triangulate, \cU)$ we have the following.

\begin{lemma}[{\cite[Corollary 15]{medina2022cube_einfty}}]\label{l:zig-zag reduced}
	For any cubical set $Y$,
	\[
	\cchains(Y) \xra{\cchains(\xi)}
	\cchains(\cubify{\triangulate Y}) \xla{\CS}
	\schains(\triangulate Y),
	\]
	is a natural zig-zag of quasi-isomorphisms of restricted $\UM$-coalgebras.
\end{lemma}

\subsection{The cobar construction as an $E_\infty$-model of Kan's loop group}

Let $X$ be a reduced simplicial set.
By \cref{c:e-ccobar and e-cobar}, the extended cobar construction $\adamsE(X)$ is isomorphic to the chains of the cubical monoid $\ccobarE(X)$.
We can use this isomorphism to transfer the $\UM$-coalgebra structure on $\cchains(\ccobarE(X))$ to $\adamsE(X)$.
Additionally, \cref{widehatgcobarandG} induces a quasi-isomorphism of $\UM$-coalgebras
\[
\schains(\triangulate\ccobarE(X)) \to \schains(\kan(X)).
\]
Combining these constructions with \cref{l:zig-zag reduced} we obtain the following zig-zag of quasi-isomorphisms relating the extended cobar construction and the chains of the Kan loop group as $E_\infty$-coalgebras.

\begin{theorem}
	For any reduced simplicial set $X$,
	\[
	\adamsE(X) \cong
	\cchains(\ccobarE(X)) \xra{\cchains(\xi)}
	\cchains(\cubify{\triangulate\ccobarE(X)}) \xla{\CS}
	\schains(\triangulate\ccobarE(X)) \to
	\schains(\kan(X))
	\]
	is a natural zig-zag of quasi-isomorphisms of restricted $\UM$-coalgebras.
\end{theorem}