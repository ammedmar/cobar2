% !TEX root = ../cobar2.tex

\section{Categorical comparison}

\subsection{Rigidification and homotopy coherent nerve}\label{ss:rigidification}

In order to relate the functors $\ccobarE \colon \sSet^0 \to \Mon_{\cSet}$ and $\kan \colon \sSet^0\to \sGrp$ we introduce a third construction, the \textit{rigidification functor}
\[
\rigid \colon \sSet^0 \to \Mon_{\sSet}.
\]
This functor is obtained by restricting to $\sSet^0$ a more general construction
\[
\rigid \colon \sSet \to \Cat_{\simplex},
\]
where $\Cat_{\simplex}$ denotes the category of small categories enriched over $\sSet$.
We now recall the construction and refer to \cite{dugger2011rigidification} for further details.

Given integers $0 \leq i \leq j$ denote by $P_{i,j}$ the category whose objects are all the subsets of $\{i, i+1, \dots, j\}$ containing both $i$ and $j$ and morphisms are inclusions.
For each integer $n \geq 0$ define $\rigid(\simplex^n) \in \Cat_{\simplex}$ to have the set $\{0, \dots, n\}$ as objects and if $i \leq j$, define $\rigid(\simplex^n)(i,j)= N(P_{i,j})$, where $N$ is the ordinary nerve functor.
If $j < i$, $\rigid(\simplex^n)(i,j) = \emptyset$.
The composition law in $\rigid(\simplex^n)$ is induced by the functor $P_{j,k} \times P_{i,k} \to P_{i,k}$ defined as the union of sets.
The assignment $[n] \mapsto \rigid(\simplex^n)$ defines a cosimplicial object in $\Cat_{\simplex}$.
For any reduced simplicial set $S$ we can now define
\[
\rigid(S) = \colim_{\simplex^n \to S} \rigid(\simplex^n).
\]

The functor $\rigid$ has a right adjoint, called the \textit{homotopy coherent nerve functor} and denoted by
\[
\nerve \colon \Cat_{\simplex} \to \sSet,
\]
whose $n$-simplices are given by
\[
\nerve(\mathcal{C})_n = \Hom_{\Cat_{\simplex}}(\rigid(\simplex^n), \mathcal{C}).
\]

The category of simplicial sets also serves as a combinatorial framework to model the homotopy theory of $\infty$-categories.
More precisely, both Joyal and Lurie independently described a model category structure on $\sSet$ whose weak equivalences are given by a sub-collection of the weak homotopy equivalences called the \textit{categorical equivalences}, cofibrations are injective maps, and fibrant objects are quasi-categories.
This model category is known as the \textit{Joyal model structure}.
The Kan--Quillen model structure is a left Bousfield localization of the Joyal model structure obtained by localizing the single morphism $\simplex^1 \to \simplex^0$.

On the other hand, Bergner constructed a model structure on $\Cat_{\simplex}$ having as weak equivalences those maps of simplicially enriched categories that induce a weak homotopy equivalence between the simplicial sets of morphisms and are essentially surjective after passing to the homotopy categories.
Fibrant objects are categories enriched over Kan complexes.
This model category is known as the \textit{Bergner model structure}.

The adjunction between rigidification and the homotopy coherent nerve yields an equivalence of homotopy theories modeling the $\infty$-categories.
More precisely, we have the following statement, which is one of the main results of \cite{dugger2011mappingspaces}.

\begin{proposition} \label{joyalbergner}
	The adjunction
	\[
	\rigid \colon \sSet \leftrightarrows \Cat_{\simplex} :\!\nerve
	\]
	induces a Quillen equivalence when $\sSet$ is equipped with the Joyal model structure and $\Cat_{\simplex}$ with the Bergner model structure.
\end{proposition}

The next result, which is an observation of the second named-author and Zeinalian, relates the cubical cobar construction and the rigidification functor via the triangulation functor as defined in \cref{ss:triangulation and its adjoint}.\todo{Please define the triangulation functor here intuitively providing a reference.}

\begin{proposition} \label{Candgcobar}
	The composition
	\[
	\triangulate \, \ccobar \colon \sSet^0 \to \Mon_{\sSet}
	\]
	is naturally isomorphic to
	\[
	\rigid \colon \sSet^0 \to \Mon_{\sSet}.
	\]
\end{proposition}

The above statement is Proposition 5.3 in \cite{rivera2018cubical}, where $\ccobarE$ is denoted by $\rigid_{\cube_c}$.
The proof relies on a description of the rigidification functor in terms of necklaces due to Dugger and Spivak \cite{dugger2011rigidification}.
\cref{Candgcobar} also holds in the many-object setting, but we do not need this level of generality for the purposes of this article.

\subsection{The rigidification functor and Kan's loop group construction}

Denote by
\[
\iota \colon \sGrp \to \Mon_{\sSet}
\]
the natural fully faithful embedding through which we may regard simplicial groups as simplicial monoids.
The starting point of the comparison between the rigidification functor and Kan's loop group functor is the following statement about their adjoints, which is Proposition 2.6.2 in \cite{hinich2007deformation}.

\begin{proposition} \label{p:hinich}
	For any simplicial group $\mathcal{G} \in \sGrp$, there is a natural weak homotopy equivalence
	\[
	\psi_{\mathcal{G}} \colon \classifying(\mathcal{G}) \xrightarrow{\simeq} \nerve\iota(\mathcal{G}).
	\]
\end{proposition}

We now relate the rigidification functor $\rigid$ and the Kan loop group functor $\kan$.
First we introduce some notation.
The embedding $\iota \colon \sGrp \to \Mon_{\sSet}$ has a left adjoint, which we denote by
\[
\mathcal{L} \colon \Mon_{\sSet} \to \sGrp.
\]
More precisely, this is the functor from simplicial monoids to simplicial groups given by formally inverting all morphisms (degree by degree) subject to the usual relations.
If $M \in \Mon_{\sSet}$ and $A \subseteq M_0$ is a subset of $0$-morphisms, we denote by $\mathcal{L}_AM$ the simplicial monoid obtained by formally inverting the elements of $A$, i.e. the pushout
\[
\mathcal{L}_AM= M \coprod_A \mathcal{L}F(A),
\]
where $F(A)$ is the monoid freely generated by the set $A$ regarded as a discrete simplicial monoid.

\begin{lemma} \label{CandG}
	There are natural weak equivalences of simplicial monoids
	\[
	\localization_{X_1} \! \rigid(X) \xra{\simeq} \localization \rigid(X) \xra{\simeq} \kan(X)
	\]
	for any reduced simplicial set $X$.
\end{lemma}

\begin{proof}
	Since $\rigid$ is a left Quillen functor and every simplicial set $X$ is cofibrant in the Joyal model structure, it follows from \cref{joyalbergner} that the simplicially enriched category $\rigid(X)$ is cofibrant.
	Hence, Proposition 9.5 of \cite{dwyer1980simplicial} implies that the natural inclusion $\localization_{X_1} \! \rigid(X) \to \localization \rigid(X)$ is a weak equivalence of simplicially enriched categories.

	By \cref{p:hinich} we have that $\psi_{\kan(X)} \colon \classifying \kan(X) \xrightarrow{\simeq} \nerve \iota(\kan(X))$ is a weak homotopy equivalence for any reduced simplicial set $X$.
	By \cref{p:kan adjuntion}, we have a weak homotopy equivalence $X \xrightarrow{\simeq} \classifying \kan(X)$ given by the unit of the adjunction.
	Composing these two maps we obtain a weak homotopy equivalence
	\[
	X \xrightarrow{\simeq} \nerve\iota(\kan(X)).
	\]
	The Quillen equivalence of \cref{joyalbergner} localizes to a Quillen equivalence
	\[
	\localization \rigid \colon \sSet^0 \leftrightarrows \sGrp \colon \nerve \iota
	\]
	when $\sSet^0$ is equipped with the Kan--Quillen model structure and $\sGrp$ with the model structure of \cref{p:kan adjuntion}.
	It follows that the adjoint of the weak homotopy equivalence $X \xra{\simeq} \nerve \iota(\kan(X))$ is a weak equivalence of simplicial groups
	\[
	\localization \rigid(X) \xra{\simeq} \kan(X),
	\]
	which finishes the proof.
\end{proof}

\subsection{Localized cubical cobar and the Kan loop group}

We have the following comparison between the triangulation of the localized geometric cobar construction and the Kan loop group.

\begin{corollary} \label{widehatgcobarandG}
	There is a natural weak equivalence of simplicial monoids
	\[
	\triangulate \ccobarE(X) \xrightarrow{\simeq} \kan(X)
	\]
	for any reduced simplicial set $X$.
\end{corollary}

\begin{proof}
	After localizing the isomorphism of \cref{Candgcobar} at the $1$-simplices, we obtain a natural isomorphism of simplicial monoids
	\[
	\triangulate \ccobarE(X) \cong \localization_{X_1} \! \rigid(X).
	\]
	Hence, the result follows from \cref{CandG}.
\end{proof}

The above statement should be understood as a lift of Hess and Tonk's quasi-isomorphism between the extended cobar construction and the chains on the Kan loop group to the level of simplicial sets.
In fact, it is explained in \cite{minichello2021path} how the weak equivalence
\[
\triangulate \ccobarE(X) \cong \mathcal{L}_{X_1} \! \rigid(X) \to \kan(X)
\]
obtained in \cref{widehatgcobarandG}
can be described explicitly in terms of the Szczarba operators used in \cite{hess2010cobar}.
More precisely, Hess and Tonk's comparison map is precisely the composition
\[
\adamsE(X) = \cchains(\ccobarE(X)) \to \schains(\triangulate\ccobarE(X)) \to \schains (\kan(X)),
\]
where the first map is determined by the shuffle quasi-isomorphism
$\cchains(\cube^n) \to \schains\!\big((\simplex^1)^{\times n}\big)$ from a cube to its triangulation and the second map is induced by the weak equivalence of \cref{widehatgcobarandG} after applying chains.